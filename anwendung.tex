% REVIEW: neue Features einfügen und ausreichend erklären. Bilder und Referenzen.
\section{Anwendung}\label{sec:features}

\subsection{Start}\label{subsec:start}

Beim Öffnen der Anwendung wird geprüft, ob sich in den Shared-Preferences des Android-Geräts bereits ein gültiges User-Token und eine passende User-Id befindet.
Ist dies der Fall wird der Benutzer anhand dieser eingeloggt.
Andernfalls öffnet sich zunächst die Login-Ansicht (siehe Abbildung~\ref{loginshopscreen}).

In der Login-Ansicht muss der Benutzer die Textfelder für den Benutzernamen und die E-Mail-Adresse ausfüllen.
Durch Aktivieren des Switch-Buttons hat der Benutzer die Möglichkeit, einen Administrator-Account zu erstellen.
Tippt der Benutzer auf den Login-Button wird ein neuer Nutzer auf Basis der in der Oberfläche eingegebenen Daten erstellt.
Das User-Token wird für die weitere Verwendung in den Shared-Preferences zwischengespeichert und die Login-Ansicht schließt sich wieder.

Sowohl im Fall des automatischen Logins, als auch nach der Anmeldung in der Login-Ansicht, öffnet sich dann die Shop-Ansicht.

% Bild vom Login-Screen mit Erklärung
\dfigure{screenshots/login.png}{screenshots/shop.png}{Login-Ansicht und Shop-Ansicht}{\label{loginshopscreen}}

\subsection{Navigation} \label{subsec:navigation}

Mittels des Kontext-Menu-Button in der oberen, linken Bildschirmecke der Shop-Ansicht oder durch Wischen nach Rechts erscheint ein Navigation Drawer.

Im oberen Bereich des Navigation Drawers befinden sich Informationen zum Benutzer (Der Nutzernamen, die Email-Adresse und das Guthaben).
Falls der Nutzer ein Administrator ist, steht hinter dem Benutzernamen in Klammern ''Admin''.
Zudem befindet sich in diesem Fall ein Stift-Button auf der rechten Seite.
Über diesen öffnet sich die Nutzer-Bearbeiten-Ansicht, der das Bearbeiten des eigene Benutzer-Profils ermöglichst.

Darunter zeigen sich viele Buttons, mit denen zu allen Ansichten der App navigiert werden kann:

\begin{itemize}
	\item Shop-Button: Die Shop-Ansicht wird geöffnet.
	Dies ist der zentrale Ansicht der App, da hier Artikel gekauft werden können.

	\item Kaufhistorie-Button: Die Kaufhistorien-Ansicht öffnet sich, in welchem alle vergangenen Einkäufe nach Datum und Zeit sortiert aufgelistet werden (siehe Abbildung ~\ref{purchases-screen}).

	\item Nutzer-Button (Administrator): Dieser Button ist nur verfügbar, wenn der Benutzer ein Administrator ist.
	Tippt er auf diesen Button, öffnet sich das Nutzer-Fragment, das die Liste aller Nutzer zeigt.

	\item Statistiken-Button: Wird dieser Button geklickt, öffnet sich der Statistiken-Ansicht, welches Statistiken zum Verhalten des Benutzers liefert.

	\item Einstellungen-Button: Über diesen Button kann die Einstellungs-Ansicht geöffnet werden.
	In diesem kann ein Nachtmodus aktiviert werden.

\end{itemize}

Der unterste Button, der mit dem Label \gqq{Logout} versehen ist, hat nichts mit der Navigation zwischen Ansichten zu tun.
Beim Tippen auf diesen Eintrag werden die in den Shared-Preferences gespeicherten Anmeldedaten gelöscht.
Der Benutzer wird zur Login-Ansicht weitergeleitet, um sich einen neuen Account zu erstellen.
Auf Grund des Designs des REST-Servers ist eine erneute Anmeldung nicht möglich.

\subsection{Shop-Ansicht}\label{subsec:shop-screen}

Nach erfolgreicher Anmeldung erfolgt eine Weiterleitung auf die Shop-Ansicht.
Dieser zeigt eine Liste der Gegenstände, die zum Kaufen verfügbar sind (siehe Abbildung~\ref{loginshopscreen}).

Ein Eintrag in dieser Liste enthält folgende Komponenten:

\begin{itemize}
	\item Den Namen des ausgewählten Gegenstands

	\item Den Preis für einen Gegenstande des ausgewählten Typs

	\item Die ausgewählte Anzahl der Gegenstände des ausgewählten Typs, welche man kaufen möchte.
	Sie erhöht sich mit jedem Tippen auf den Eintrag.
	Wischt man hingegen von rechts nach links über den Eintrag, so wird die ausgewählte Anzahl auf 0 gesetzt.
	Die Anzahl wird nur angezeigt, wenn sie größer als 0 ist.

	\item Der Gesamtpreis für den Gegenstandstyp.
	Hierbei handelt es sich um den Kumulierten Preis (ausgewählte Anzahl $\cdot$ Preis) des Gegenstands.
	Dieser wird nur angezeigt, wenn mehr als ein Gegenstand des ausgewählten Typs verlangt wird.

	\item verfügbare Anzahl: Zeigt an, wie viele Gegenstände des ausgewählten Typs gekauft werden können.
	Sie kann aufgrund des Design der REST-API negativ werden.
\end{itemize}

Die Liste ist nach der Art der Gegenstände sortiert (Beispiel: Wasser, Saft, \ldots).
Als Überschrift ist jeder Gruppierung von Artikeln derselben Art ein Listeneintrag vorangestellt, der den Namen der Art trägt.
Tippt der Benutzer auf eine Überschrift, faltet sich die Kategorie zusammen.
Bei erneutem Tippen werden die Artikel wieder angezeigt.

Auf dieselbe Art funktionieren alle Listen der App mit Kategorien.

\subsubsection{Artikel zum Einkaufswagen hinzufügen} \label{subsubsec:shoppingcart-add-item}

Tippt der Benutzer auf einen Listeneintrag, z.B.\ Zitronenlimonade, wird ein Exemplar in den Warenkorb gelegt.
Der Benutzer kann einen Listeneintrag so oft hinzufügen, wie Exemplare verfügbar sind.
Die Anzahl, der von einem Typ zum Einkaufswagen hinzugefügten Exemplare wird auf dem Listeneintrag abgebildet.
Sind zwei oder mehr Exemplare eines Artikels im Warenkorb, wird der Gesamtpreis der ausgewählten Exemplare des Artikels auf dem Listeneintrag abgebildet.

\subsubsection{Artikel aus dem Einkaufswagen entfernen} \label{subsubsec:shoppingcart-del-item}

Artikel können aus dem Einkaufswagen entfernt werden, indem der entsprechende Listeneintrag nach links gewischt wird.
Dabei zeigt sich unter dem Listeneintrag ein Mülleimer-Symbol auf dunklem Grund, welches den Vorgang verdeutlicht.
Es werden dann alle Exemplare des Artikels aus dem Warenkorb entfernt.

\subsubsection{Verhalten des Einkaufswagen-Buttons} \label{subsubsec:shoppingcart-button}

Der Einkaufswagen-Button schwebt über der Artikel-Liste.
Durch das Tippen auf diesen kann man den Einkaufswagen in einer separaten Ansicht betrachten und den Kauf tätigen.

Der Button zeigt sich erst, sobald sich mindestens ein Element im Warenkorb befindet.
Werden alle Elemente aus dem Warenkorb entfernt, verschwindet der Button wieder.

Befindet sich mehr als ein Element im Warenkorb, verschwindet der Einkaufswagen-Button temporär, sobald der Benutzer nach unter scrollt, damit untere Listenelemente nicht vom Button verdeckt werden.
Scrollt der Benutzer nach oben, erscheint der Einkaufswagen-Button wieder.

\subsubsection{Artikel kaufen} \label{subsubsec:shop-buy}

Wenn alle gewünschten Gegenstände ausgewählt sind, wird durch Tippen auf den Einkaufswagen-Button die Warenkorb-Ansicht geöffnet.
Hierbei handelt es sich um eine Darstellung nur der ausgewählten Gegenstände mit der bereits bekannten Eintragstruktur.
Auch hier können durch Tippen auf die Einträge Gegenstände hinzugefügt und durch Wischen über einen Eintrag alle Gegenstände dieses Typs entfernt werden.
Mit dem Löschen-Button kann der komplette Einkaufswagen geleert werden.
Durch Tippen auf den Submit-Button (Haken) erscheint ein Modal-Fenster (siehe Abbildung~\ref{shoppingCartCheck}) mit einer Abfrage, ob der Kauf durchgeführt werden soll.
Hier wird nochmals der Gesamtpreis des Einkaufs angezeigt.
Tippen auf \gqq{Ja} bewirkt, dass der Kauf getätigt wird und man auf die Shop-Ansicht weitergeleitet wird.
Durch ein Pop-Up am unteren Bildschirmrand wird gezeigt, dass der Kauf erfolgreich war (siehe Abbildung~\ref{purchaseSuccess}).

% Bild des Bestätigungs-Modal, nachdem Haken im Shopping-Cart gecklickt
% optional: Bild von Benachrichtigung nachdem Kauf erolgreich
%\dfigure{figures/screen_placeholder.png}{figures/screen_placeholder.png}{Letzte Bestätigung vor Kauf und Pop-Up nach erfolgreichem Einkauf}{\label{shoppingCartCheck}\label{purchaseSuccess}}

\subsubsection{Barcode von Artikeln scannen} \label{subsubsec:shop-scan-item}

% Was passiert, wenn ein normaler Nutzer den Barcode eines Artikels einscannt
Durch Tippen auf das Kamera-Icon in der oberen rechten Ecke der Shop-Ansicht, das in Abbildung~\ref{mainscreen} zu sehen ist, öffnet sich die Kamera des Gerätes mit einem eingebauten Barcode-Scanner, wie in Abbildung~\ref{barcodeScanner}.
Hiermit kann der Benutzer Barcodes einscannen und die Artikel bequemer erwerben.

\dfigure{screenshots/barcode-scan.png}{screenshots/barcode-add-to-cart}{Barcode Scanner}{\label{barcodeScanner}}

\subsection{Shop-Ansicht (für Administratoren)} \label{subsec:shop-screen-admin}

Ist der Benutzer ein Administrator, ist in der Toolbar ein Stift-Symbol zu sehen.
Mit diesem lässt sich der Bearbeitungs-Modus aktivieren.
Dann verschwindet in der Toolbar das Artikel-Scannen-Symbol und wird ersetzt durch ein Plus-Symbol.
Außerdem färbt sich die Toolbar, um zu signalisieren, dass sich der Benutzer im Bearbeitungs-Modus befindet.
Tippt der Benutzer erneut auf das Stift-Symbol, wird der Bearbeitungs-Modus deaktiviert.
Die Toolbar wird wieder blau gefärbt und das Plus-Symbol wird wieder durch das Artikel-Scannen-Symbol ersetzt.

\subsubsection{Artikel bearbeiten} \label{subsubsec:shop-edit-items}

Befindet sich der Benutzer im Bearbeitungs-Modus und tippt auf einen Listeneintrag so öffnet sich, wie es in Abbildung~\ref{createEditItem} rechts zu sehen ist, die Artikel-Bearbeiten-Ansicht.
Der Benutzer hat nun die Möglichkeit die folgenden Eigenschaften des Artikels zu ändern:

\begin{itemize}
	\item Der Name des zu erstellenden Gegenstands

	\item Der Preis des zu erstellenden Gegenstands

	\item Die Menge, in der der Gegenstand verfügbar ist

	\item Die Art des Gegenstands.
	Hier wird aus einem Drop-Down-Menu eine Art ausgewählt.
\end{itemize}

Der Barcode selbst kann nicht bearbeitet werden.
Tippt der Benutzer auf den Bearbeiten-Button, wird er zur Shop-Ansicht im Bearbeitungs-Modus weitergeleitet, der den bearbeiteten Artikel enthält.

\dfigure{screenshots/item-create.png}{screenshots/item-edit.png}{Anlegen eines neuen Artikels / Bestehenden Artikel bearbeiten}{\label{createEditItem}}

\subsubsection{Artikel hinzufügen} \label{subsubsec:shop-add-items}

Ist der aktuelle Benutzer als Administrator eingeloggt, kann dieser mit dem + Knopf im oberen Banner der Shop-Ansicht, wie in Abbildung~\ref{createEditItem} links demonstriert, neue Artikel hinzufügen.
Dazu müssen für jeden neuen Artikel die folgenden Daten eingegeben werden:

\begin{itemize}
	\item Der Barcode des Artikels.
	Anstatt selbst einen Barcode zu wählen, kann durch Tippen des ID-Generieren-Buttons eine zufälliger ID generiert werden.

	\item Der Name des neuen Artikels.

	\item Der Preis des neuen Artikels.

	\item Die Menge, in der dieser Artikel auf Vorrat ist.

	\item Die Warengruppe des Artikels.
	Hier wird aus einem Drop-Down-Menu eine Art ausgewählt.
\end{itemize}

Wird auf den Anlegen-Button getippt so wird der Artikel erstellt.
Es öffnet sich wieder die Shop-Ansicht im Bearbeitungs-Modus, dessen Artikel-Liste den neu erstellten Artikel enthält.

\subsubsection{Barcode von Artikeln scannen} \label{subsubsec:shop-admin-scan-item}

% Was passiert, wenn ein Artikel gescannt wird und der Nutzer ein Administrator ist
Wird ein Barcode von einem Administrator gescannt so wird die App nach schauen ob es zu diesem Barcode bereits einen Artikel gibt oder nicht.
Ist dieser Artikel bereits im Sortiment so kann der Administrator diesen wie aus Abbildung~\ref{barcodeScanner} gewohnt einfach erwerben.
Ist dieser Artikel jedoch noch nicht im Sortiment wird dem Administrator, wie in Abbildung~\ref{barcodeScanner-createItem} geschildert, nun die Möglichkeit angeboten diesen Artikel dem Sortiment hinzuzufügen.

\cfigure{screenshots/barcode-item-create.png}{Barcode Scanner - Einfach neue Artikel anlegen}{\label{barcodeScanner-createItem}}

\subsection{Kaufhistorien-Ansicht} \label{subsec:purchases-screen}

In der Kaufhistorien-Ansicht sieht der Benutzer alle seiner getätigten Transaktionen in einer Liste (siehe Abbildung~\ref{purchases-screen}).

Ein Listeneintrag stellt einen oder mehrere Artikel einer Sorte dar, die in der Transaktion gekauft wurden.
Er enthält folgende Komponenten:

\begin{itemize}
	\item Den Namen des Artikels

	\item Den Barcode des Artikels

	\item Die Anzahl, die man von diesem Artikeltyp gekauft hat.

	\item Der Gesamtpreis, der für diesen Artikeltyp anfällt.
	Hierbei handelt es sich um den Kumulierten Preis (ausgewählte Anzahl $\cdot$ Preis) des Artikels.
	Dieser wird nur angezeigt, wenn mehr als ein Artikel des Typs gekauft wurde.
\end{itemize}

Die Listeneinträge einer Transaktion sind nach dem Datum der Transaktion gruppiert.

% Bild des Purchases-Screen, evtl. Erklärung anpassen
\cfigure{screenshots/purchase-history.png}{Liste getätigter Einkäufe}{\label{purchases-screen}}

\subsection{Nutzer-Bearbeiten-Ansicht} \label{subsec:edit-user-screen}

In der Nutzer-Bearbeiten-Ansicht kann der Benutzer sein eigenes Profil bearbeiten (siehe Abbildung~\ref{userScreen}).

\dfigure{screenshots/user-edit.png}{screenshots/user-list.png}{Nutzer-Bearbeiten-Ansicht und Nutzer-Ansicht}{\label{userScreen}}

Er kann die folgenden Daten verändern:

\begin{itemize}
	\item Den Nutzernamen
	\item Die Email-Adresse
	\item Das eigene Guthaben, falls er ein Administrator ist
\end{itemize}

Tippt der Benutzer auf den Speicher-Button, wird das Nutzer-Profil gespeichert und es öffnet sich die Nutzer-Ansicht.

\subsection{Nutzer-Ansicht (für Administratoren)} \label{subsec:user-screen}

Die Nutzer-Ansicht ist nur im Navigation-Drawer zu sehen, falls der Benutzer ein Administrator ist.

In der Nutzer-Ansicht hat der Benutzer einen Überblick über alle Nutzer, die beim Server angemeldet sind (siehe Abbildung~\ref{userScreen}).
Sie sind nach den Anfangsbuchstaben ihrer Benutzernamen gruppiert und alphabetisch geordnet.

Tippt der Benutzer auf einen Listeneintrag, öffnet sich die Nutzer-Bearbeiten-Ansicht, die in Abschnitt~\ref{subsec:edit-user-screen} erläutert wurde.
Statt des eigenen Benutzer-Accounts, kann der Account des ausgewählten Benutzers bearbeitet werden.

\subsection{Statistiken-Ansicht} \label{subsec:statistics-screen}

Die Statistiken-Ansicht ist über die Navigation erreichbar.
Diese stellt den Administratoren als auch den Benutzern der App unterschiedliche Informationen grafisch dar.
Ein Administrator kann dort einsehen welche Artikel wie oft gekauft wurden und somit einen Überblick über die verkauften Stückzahlen für jeden Artikel erhalten.
Des weiterem wird der Betrag von allen verkauften Artikeln, der Betrag von den selbst eingekauften Artikeln, die Anzahl aller verkaufter Artikel und die Anzahl der selbst eingekauften Artikel dargestellt.
Die Ansicht eines Administrators ist in Abbildung~\ref{statistics-admin} zu sehen.
Ein Benutzer erhält lediglich einen Überblick über seine eignen Einkäufe.
Dabei kann der Benutzer sehen wie viele Exemplare eines Artikels dieser gekauft hat.
Des weiterem wird dem Benutzer dargestellt wie viel Geld dieser bereits ausgegeben hat und wie viele Artikel dieser insgesamt schon gekauft hat.
Die Ansicht eines Benutzers ist ebenfalls in Abbildung~\ref{statistics-user} dargestellt.
Natürlich kann ein Administrator, wie der Benutzer, auch seine eignen Einkäufe einsehen.
In den Statistiken werden alle Beträge in Euro angezeigt.

\dfigure{screenshots/statistics-all.png}{screenshots/statistics-mine.png}{Statistiken - Ansicht des Administrators und Ansicht des Benutzers}{\label{statistics-admin}\label{statistics-user}}

\subsection{Einstellungen-Ansicht} \label{subsec:settings-screen}

In der Einstellungen-Ansicht hat der Benutzer die Möglichkeit, durch das Tippen eines Switch-Buttons den Nachtmodus der App zu aktivieren.
Die App erscheint dann in einem dunkleren Design und ist dadurch schonender für die Augen.
Bei erneutem Tippen des Switch-Buttons wird die App wieder in helleren Farben angezeigt.
Ob der Nachtmodus aktiviert oder deaktiviert ist, wird in den Shared-Preferences gespeichert.

\subsection{Aktualisieren von Server-Daten}\label{subsec:aktualisieren-von-server-daten}

Alle Ansichten, die Daten vom Server in einer Liste darstellen, können aktualisiert werden.
Dazu gehören die Shop-Ansicht, die Nutzer-Ansicht, sowie die Kaufhistorien-Ansicht.
Eine Aktualisierung erfolgt, indem am oberen Ende der Liste ein Zieh-Bewegung nach Unten durchgeführt wird.
Dabei wird ein Lade-Balken sichtbar, der den Fortschritt der Aktualisierung anzeigt.

\subsection{Unterstützte Sprachen} \label{subsec:languages}

Von der App wird derzeit Englisch und Deutsch unterstützt.
Die Wahl der Sprache hängt hierbei von der Systemsprache des Smartphones ab.

\subsection{Shortcut-Funktion zum Scannen von Artikeln} \label{subsec:scan-item-shortcut}

Mit Android-Shortcuts können beliebte Funktionen einer App direkt im Android-Startbildschirm aufgerufen werden, ohne durch die App navigieren zu müssen.
Die Shortcuts, die dem Benutzer bei einer App angeboten werden, können durch langes Drücken des App-Icons angezeigt werden.
Bei der hier vorgestellten App erscheint in diesem Fall ein Shortcut zum Scannen von Artikeln.
Tippt der Benutzer auf den angezeigten Shortcut, öffnet sich die App sofort in der Ansicht zum Scannen von Artikeln, wie in Abschnitt~\ref{subsubsec:shop-scan-item} beschrieben.
Im weiteren Verlauf verhält sich die App, als hätte man die App geöffnet und zur besagten Ansicht navigiert.
