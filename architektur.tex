\section{Architektur}\label{sec:architecture}

\subsection{Datenmodell}\label{subsec:datamodel}

Das Datenmodell umfasst drei Klassen: \texttt{Item}, \texttt{User} und \texttt{Purchase}.
Die Klassen sind passend zur bereitgestellten Datenbank implementiert und haben die in Abbildung~\ref{fig:datamodel} gezeigte Struktur.

\dfigure{figures/datamodel.png}{figures/main-activity-aufbau.png}{Datenmodell und Aufbau der MainActivity}{\label{fig:datamodel}\label{fig:main-activity-aufbau}}

\subsection{Frontend-Architektur}\label{subsec:frontend}
% generelle Architektur (Klassen, verwendete Patterns, Fragments, Activities) und deren grobe Erklärung

Die App nutzt eine Single-Activity-Architektur und läuft damit hauptsächlich in einer einzigen Activity, der \texttt{MainActivity}.
Durch diese Architektur ergeben sich eine Vielzahl von Vorteilen.

\subsubsection{Aufbau der MainActivity}

Der Aufbau der \texttt{MainActivity} wird in Abbildung~\ref{fig:main-activity-aufbau} dargestellt.
Er besteht aus einem Frame-Layout und einer Toolbar.
Zudem wird durch eine Wisch-Bewegung nach Rechts ein Navigation-Drawer angezeigt.

\subsubsection{Fragmente}

Ansichten wurden durch Fragmente realisiert.
Dabei handelt es sich um Container, die alle UI-Komponenten der jeweiligen Ansicht enthalten.
Für jede Ansicht, die in Abschnitt~\ref{sec:features} beschrieben wurde, existiert ein solches.
Befindet sich der Benutzer zum Beispiel in der Shop-Ansicht, wird das Shop-Fragment im Frame-Layout angezeigt.
Wechselt der Benutzer zu einer anderen Ansicht, wird das Fragment im Frame-Layout durch das Fragment der neuen Ansicht ersetzt.
Dies erfolgt durch Zugriff auf den \texttt{FragmentManager} der \texttt{MainActivity}.

% TODO Vorteile der Fragmente gegenüber Activities

\subsubsection{Toolbar}

Im Gegensatz zu den Fragmenten, wird die Toolbar beim Wechsel der Ansicht nicht ausgetauscht, sondern kann vom neu angezeigten Fragment manipuliert werden.

Die Voraussetzung dafür ist, dass alle Fragmente von \texttt{KitchenFragment} erben.
Dieses stellt eine Methode bereit, die aufgerufen wird, sobald das Fragment in der \texttt{MainActivity} angezeigt wird.
Beispielsweise können hier spezielle Symbole des Fragments in der Toolbar angezeigt werden.

Hier zeigt sich ein Vorteil der Single-Activity-Architektur.
Wäre jede Ansicht eine eigene Activity, müsste jedes Layout eine eigene Toolbar beinhalten.
Die daraus resultierende Redundanz kann hier vermieden werden, da nur eine Toolbar im Layout der \texttt{MainActivity} benötigt wird.

\subsubsection{Navigation-Drawer}

Durch die Single-Activity-Architektur wird das Benutzen eines Navigation-Drawers erst möglich, da einzig im Layout der \texttt{MainActivity} ein Navigation-Drawer eingebunden werden muss.
Dies führt dazu, dass das Tippen eines Elements im Navigation-Drawer zentral verarbeitet werden kann.
Dadurch kann beispielsweise das Zurückgleiten des Navigation-Drawers nach einem Tipp-Event sehr einfach erreicht werden.

Im Gegensatz dazu müsste bei der Verwendung vieler Activities, jede Activity ein Navigation-Drawer in deren Layout definieren.
Dies wäre nicht nur redundant, sondern würde auch die Verarbeitung von Tipp-Events und die korrekte Anzeige des Navigation-Drawers weitaus komplexer machen, da mit anderen Activities kommuniziert werden müsste.

\subsubsection{Ausnahmen}

Weil es sich angeboten hat und man sich idealerweise nur einmal anmeldet, ist die Login-Ansicht eine eigene Activity.
Auch der BarcodeScanner ist eine eigene Activity, da die \texttt{BarcodeScanner}-Library dies verlangt.

\subsubsection{Implementierung der Listen}

In weiten Teilen der App werden Daten in Listen angezeigt.
Dafür wurde die ressourcensparende Listen-Klasse \texttt{RecyclerView} verwendet.
Zur Vermeidung von Redundanz, wurde die in Abbildung~\ref{fig:recyclerview-diagram} gezeigte Architektur implementiert.
Sie ist am Beispiel des Shop-Fragments dargestellt.

Der Kern dieser Architektur ist das \texttt{GeneralRecyclerView}.
Ein Fragment, das eine Liste in Form eines \texttt{GeneralRecyclerView} anzeigt, implementiert das Interface \texttt{RecyclerViewEventHandler} und wird dadurch bei auftretenden Events benachrichtigt.
Wenn beispielsweise die Liste im Shop-Fragment nach Unten gescrollt wird, erfährt es dies über ein Scroll-Event vom \texttt{GeneralRecyclerView}.
Das Shop-Fragment lässt darauf folgend den Einkaufswagen-Button verschwinden.

Damit das \texttt{GeneralRecyclerView} unabhängig ist von den Daten, die es darstellt, wird es über das \texttt{RecyclerController}-Interface mit dem \texttt{ItemRecyclerController} verbunden.
Bei einem auftretenden Event, benachrichtigt es den \texttt{ItemReyclerController}, der Änderungen an Daten über das Backend vornimmt.
Führt der Benutzer beispielsweise eine Wisch-Bewegung im Shop-Fragment durch, um einen Artikel zu löschen, wird der \texttt{ItemRecyclerController} benachrichtigt und löscht den Artikel aus dem Datenmodell.

Das im Beispiel verwendete Shop-Fragment, kann durch jedes beliebige anderen Fragment ersetzt werden und ermöglichst damit eine einfache Einbindung des \texttt{RecyclerViews} in jedes Fragment.
Um ein \texttt{RecyclerView} für eine neue Art von Daten zu implementieren, muss nur ein neuer \texttt{RecyclerController} geschrieben werden.
Damit können \texttt{RecyclerViews} auf einfache Art, neuen Daten zugänglich gemacht werden.

\cfigureLarge{figures/recyclerview-diagram.png}{RecyclerView Architektur am Beispiel des Shop-Fragments}{\label{fig:recyclerview-diagram}}

\subsubsection{Kommunikation mit dem Backend-Layer}

Die Architektur ermöglicht sehr einfache Kommunikation mit dem Backend-Layer.
Über statische Instanzen von Hilfsklassen haben alle UI-Klassen kontrollierten Zugriff auf das Backend.
Diese Hilfsklassen werden in Abschnitt~\ref{subsubsec:access-layer} genauer beschrieben.

\subsection{Backend-Architektur}\label{subsec:backend}

Die Backend-Architektur besteht aus drei Ebenen.
Auf der \emph{Base-Ebene} sind primitive HTTP-Funktionen implementiert,
welche von der \emph{Abstraction-Ebene} im Kontext der Kitchen-API verwendet werden.
Auf die \emph{Access-Ebene} wird direkt aus dem UI zugegriffen, dabei implementiert sie abstrakte Logik der verschiedenen clientseitigen Interaktionen.
Abbildung~\ref{fig:backend} zeigt diese Ebenen und die zugehörigen Klassen, welche im Folgenden kurz beschrieben werden.

\cfigureLarge{figures/backend.png}{Backend Architektur}{\label{fig:backend}}

\subsubsection{Base-Layer}

Die unterste Ebene bildet die Klasse \texttt{OkHttpConnection}, welche über das Interface \texttt{HttpConnection} implementiert ist.
Die Klasse ist verantwortlich für den primitiven Netzwerkverkehr per HTTP mit beliebigen Inhalten.
Es werden die HTTP-Methoden \texttt{GET}, \texttt{POST}, \texttt{PUT} und \texttt{DELETE} implementiert.
Bei fehlerhaften Anfragen (HTTP-Code $\neq$ 2xx), sowie bei fehlerhaften URLs, wird eine Exception geworfen.

\subsubsection{Abstraction-Layer}

Auf der zweiten Ebene befinden sich die Klassen \texttt{KitchenConnection} und \texttt{JsonTranslator}.

\begin{itemize}
	\item  \texttt{KitchenConnection} ist verantwortlich für die spezialisierte Kommunikation mit der bereitgestellten REST-API. Die Klasse enthält Attribute für die Basis-URL des REST-Servers, sowie für Schlüssel zur Erzeugung von Nutzern.
	Alle Funktionen der REST-API werden hier implementiert, indem jede Methode der Klasse einen Pfad der REST-API abdeckt.
	Die Funktionsnamen entsprechen der Spezifikation der REST-API\@.

	\item \texttt{JsonTranslator} ist verantwortlich für die bidirektionale Umwandlung zwischen Objekten des Datenmodells und JSON-Strings.
	Die verwendete Bibliothek für die Umwandlung ist Google's \texttt{GSON}.
\end{itemize}

\subsubsection{Access-Layer} \label{subsubsec:access-layer}

Die dritte Ebene beinhaltet fünf Klassen \texttt{Users}, \texttt{Items}, \texttt{Purchases}, \texttt{Cart} und \texttt{Session}.
Sie sind die Einstiegspunkte des Backends und stellen alle Operationen zur Verfügung, welche die Oberfläche benötigt.

\begin{itemize}
	\item \texttt{Users} bietet Methoden zum Lesen, Verändern und Löschen von Benutzern sowie einen lokalen Zwischenspeicher.
	\item \texttt{Items} bietet ähnliche Funktionalität für Artikel, erlaubt jedoch auch das Erzeugen.
	\item \texttt{Purchases} ermöglicht das Lesen und Gruppieren von Käufen aller Benutzer und des aktuell angemeldeten Benutzers.
	\item \texttt{Cart} abstrahiert eine Liste von \texttt{Purchase}-Objekten, die die aktuell im Einkaufswagen liegenden Artikel und deren Anzahl speichern.
	Hier werden auch Preisberechnungen und Bestandsprüfungen implementiert.
	\item \texttt{Session} implementiert die Logik für Login und Registrierung sowie das persistente Lesen und Speichern von Benutzer-ID und -Token.
	Weiterhin können damit einige Informationen über den aktuell angemeldeten Benutzer ermittelt werden.
\end{itemize}

Instanzen dieser Klassen sind für das UI durch statische Felder zugänglich.
Untereinander verwenden die Klassen Instanzvariablen der jeweils anderen Klassen, welche per Konstruktor gesetzt werden und somit das Testen vereinfachen.
So hat beispielsweise die Klasse \texttt{Cart} die Felder \texttt{KitchenConnection connection}, \texttt{Session session} und \texttt{Items items}, da ihre Implementierung diese benötigt.
Die statischen Feldern werden nur für Zugriff vom UI verwendet, intern greifen die Klassen nur auf die Instanzvariablen zu.

Die Methoden der Klassen sind stets in drei Bereiche unterteilt, um eine interne Struktur zu erreichen und die Auswirkungen leicht verständlich zu machen:

\begin{itemize}
	\item \texttt{Access} erlaubt den Zugriff auf den lokalen Zwischenspeicher, inklusive in gefilterter und gruppierter Form (z.B.\ eigene Käufe, Artikel gruppiert nach Kategorie).
	\item \texttt{Modification} enthält Methoden, die den lokalen Zwischenspeicher verändern.
	Dazu gehören beispielsweise das Speichern von Datenmodellobjekten und das Hinzufügen von Artikeln zum Warenkorb.
	\item \texttt{Communication} umfasst die Methoden, welche eine Serverkommunikation durchführen.
	Beispiele sind die Aktualiserung aller Benutzer und das Erzeugen oder Bearbeiten von Artikeln.
	Hier wird jedoch nicht zwischen Lesen und Schreiben getrennt.
\end{itemize}

\subsubsection{Tests}

Alle Backend-Klassen sind durch Unit-, Integrations- und Akzeptanz-Tests ausreichend getestet.
