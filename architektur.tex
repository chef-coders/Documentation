\section{Architektur}\label{sec:architecture}

\subsection{Datenmodell}\label{subsec:datamodel}

Das Datenmodell umfasst drei Klassen: \texttt{Item}, \texttt{User}, \texttt{Purchase}.
Die Klassen sind passend zur bereitgestellten Datenbank implementiert und haben die in Abbildung ~\ref{datamodel} gezeigte Struktur.

\cfigure{figures/datamodel.png}{Datenmodell}{\label{datamodel}}

\subsection{Frontend-Architektur}\label{subsec:frontend}
% generelle Architektur (Klassen, verwendete Patterns, Fragments, Activities) und deren grobe Erklärung

\subsection{Backend-Architektur}\label{subsec:backend}

Die Backend-Architektur besteht aus drei Ebenen (siehe Abbildung ~\ref{backendArchitecture}).

\subsubsection{Base-Layer}

Die unterste Ebene bildet die Klasse \texttt{OkHttpConnection}, welche über das Interface \texttt{HttpConnection} implementiert ist.
Die Klasse ist verantwortlich für den primitiven Netzwerkverkehr per HTTP mit beliebigen Inhalten.
Es werden die HTTP-Methoden \texttt{GET}, \texttt{POST}, \texttt{PUT} und \texttt{DELETE} implementiert.
Bei fehlerhaften Anfragen (HTTP-Code $\neq$ 2xx), sowie bei fehlerhaften URLs wird eine Exception geworfen.

\subsubsection{Abstraction-Layer}

Auf der zweiten Ebene befinden sich die Klassen \texttt{KitchenConnection}, \texttt{LocalDataStore} und \texttt{JsonTranslator}.

\begin{itemize}
	\item  \texttt{KitchenConnection} ist verantwortlich für die spezialisierte Kommunikation mit der bereitgestellten REST-API. Die Klasse enthält Attribute für die Basis-URL des REST-Servers, sowie für Schlüssel zur Erzeugung von Nutzern.
	Alle Funktionen der REST-API werden hier implementiert, indem jede Methode der Klasse einen Pfad der REST-API abdeckt.
	Die Funktionsnamen entsprechen der Spezifikation der REST-API\@.

	\item \texttt{LocalDataStore} ist verantwortlich für das Zwischenspeichern der User, Items und Purchases.
	Dies ermöglicht schnelleren Zugriff auf die jeweiligen Objekte.

	\item \texttt{JsonTranslator} ist verantwortlich für die bidirektionale Umwandlung zwischen Objekten des Datenmodells und JSON-Strings.
	Die verwendete Bibliothek für die Umwandlung ist Google's \texttt{GSON}.
\end{itemize}

\subsubsection{Access-Layer}

Die Dritte Ebene beinhaltet nur die Klasse \texttt{KitchenManager}.
Sie ist der Einstiegspunkt des Backend und stellt so alle Operationen zur Verfügung, welche die Oberfläche benötigt. \texttt{KitchenManager} bildet eine vollständige Abstraktion der REST-API, über die Klasse \texttt{KitchenConnection}, welche bereits die Adressen und benötigten Header der API-Aufrufe abstrahierte.

\cfigure{figures/classStructure.png}{Backend Architektur}{\label{backendArchitecture}}

\subsubsection{Tests}

Alle Backend-Klassen sind durch Unit-, Integrations- und Akzeptanz-Tests ausreichend getestet.
