\section{Einleitung}\label{sec:einleitung}

\subsection{Ziele}\label{subsec:ziele}

Ziel des Projektes war es, im Rahmen der Veranstaltung \glqq Code Camp 1: Context Awareness\grqq{} eine Android-basierte Anwendung zur Verwaltung und Aufzeichnung der Einkäufe, welche in der ComTec-Küche getätigt werden, zu erstellen.

\subsection{Funktionsweise}\label{subsec:funktionsweise}

Auf Basis einer Datenbank können Gegenstände angelegt, gekauft und gelöscht werden.
Dazu benötigte Nutzer werden ebenfalls in der Datenbank gespeichert.
Der Zugriff auf diese erfolgt über eine REST-API, welche alle notwendigen Operationen bereitstellt.

\subsection{Überblick}\label{subsec:überblick}

Nutzer erhalten bei der Registrierung ein Token, welches für den Zugriff über die REST-API benötigt wird.
Dieses wird in der Anwendung zwischengespeichert, sodass ein Nutzer mit gültigem Token automatisch angemeldet wird, wenn die Anwendung startet. \\
Der Fokus der Anwendung liegt auf dem Kauf von Gegenständen.
Dies ist auf zwei Arten möglich:

\begin{enumerate}
	\item regulär in der App: Über eine Liste der verfügbaren Gegenstände können solche per Klick zum Warenkorb hinzugefügt werden.

	\item über einen eingebetteten Barcode-Scanner können Gegenstände, deren Barcode bereits in der Datenbank registriert ist dem Warenkorb hinzugefügt werden.
\end{enumerate}

Es ist dem Nutzer zudem möglich eine Statistik zur Kaufhäufigkeit der registrierten Gegenstände einzusehen. \\
Zwecks Verwaltung ist es als Administrator ausgezeichneten Nutzern möglich neue Gegenstände wahlweise mit automatisch generierter ID oder dem Barcode-Scanner anzulegen.

% REVIEW: Update, wenn neue Sections hinzugefügt werden
\subsection{Inhalt}\label{subsec:inhalt}

In Abschnitt ~\ref{sec:features} werden die von der App bereitgestellten Features vorgestellt.
Abschnitt ~\ref{sec:architecture} erklärt die Anwendungsstruktur aus technischer Sicht, jeweils für Datenmodell (Abschnitt ~\ref{subsec:datamodel}), die Oberflächen-Architektur (Abschnitt ~\ref{subsec:frontend}) und die zugrunde liegende Grundstruktur inklusive Netzwerkverkehr (Abschnitt ~\ref{subsec:backend}).
