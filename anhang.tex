\section{Anhang}\label{sec:bib}

\subsection{Bibliotheken}\label{subsec:bibliotheken}

\begin{table}[h]
    \renewcommand{\arraystretch}{1.1}

    \begin{tabularx}{\textwidth}{|X|l|l|l|l|}
        \hline
        \textbf{Name} & \textbf{Entwickler} & \textbf{Verwendung} & \textbf{Version} & \textbf{Lizenz} \\
        [0.5ex] \hline
        Android Support Library V4 & AOSP & Android & 28.0.0 & Apache 2.0\\   % com.android.support:support-v4:28.0.0
        \hline

        Play Services Vision & Google & Play Services & 11.0.2 & Android SDK\\  % com.google.android.gms:play-services-vision:11.0.2
        \hline

        Android AppCompat Library V7 & AOSP & Benutzeroberfläche & 28.0.0 & Apache 2.0\\    % com.android.support:appcompat-v7:28.0.0
        Android ConstraintLayout & AOSP & Benutzeroberfläche & 1.1.3 & Apache 2.0\\   % com.android.support.constraint:constraint-layout:1.1.3
        Material Components For Android & AOSP & Benutzeroberfläche & 28.0.0 & Apache 2.0\\  % com.android.support:design:28.0.0
        SectionedRecyclerViewAdapter & Gustavo Pagani & Benutzeroberfläche & 1.2.0 & MIT\\   % io.github.luizgrp.sectionedrecyclerviewadapter:sectionedrecyclerviewadapter:1.2.0
        Barcode-Reader & Ravi & Benutzeroberfläche & 1.1.5 & BSD 3\\   % info.androidhive:barcode-reader:1.1.5
        MPAndroidChart & Philipp Jahoda & Benutzeroberfläche & 3.1.0 & Apache 2.0\\  % com.github.PhilJay:MPAndroidChart:v3.1.0-alpha
        \hline

        fulib & Albert Zündorf & Datenmodell & 1.0.+ & -\\    % org.fulib:fulib:1.0.+
        \hline

        OkHttp & Square & Netzwerk & 3.13.1 & Apache 2.0\\   % com.squareup.okhttp3      https://mvnrepository.com/artifact/com.squareup.okhttp3/okhttp
        Gson & Google & Netzwerk & 2.8.5 & Apache 2.0\\   % com.google.code.gson      https://mvnrepository.com/artifact/com.google.code.gson/gson
        \hline

        JUnit & JUnit & Tests & 4.12 & EPL 1.0\\   % https://mvnrepository.com/artifact/junit/junit
        JMock 2 Core & jMock.org & Tests & 2.9.0 & BSD\\    % https://mvnrepository.com/artifact/org.jmock/jmock
        \hline

        Android Testing Support Library & AOSP & Android-Tests & 1.0.2 & Apache 2.0\\   % com.android.support.test:runner:1.0.2
        Espresso Core & AOSP & Android-Tests & 3.0.2 & Apache 2.0\\    % com.android.support.test.espresso:espresso-core:3.0.2
        Espresso Intents & AOSP & Android-Tests & 3.0.2 & Apache 2.0\\ % com.android.support.test.espresso:espresso-intents:3.0.2
        Test Regeln & AOSP & Android-Tests & 1.0.2 & Apache 2.0\\  % com.android.support.test:rules:1.0.2
        \hline
    \end{tabularx}

    \caption{Übersicht der verwendeten Bibliotheken}
    \label{tab:libraries}
\end{table}

In Tabelle~\ref{tab:libraries} sind alle verwendeten Bibliotheken aufgelistet.
In den folgenden Abschnitten wird auf die Verwendung der einzelnen Bibliotheken näher eingegangen.

\subsubsection{Android} \label{subsubsec:android-libraries}
Die vierte Version der Android Support Bibliothek stellt eine große Anzahl von zusätzlichen Frameworks bereit, die für die App benötigt werden.
Beispielsweise wurden mit ihr die Fragmente, sowie der Navigation-Drawer realisiert.
Außerdem bildet sie die Voraussetzung für die siebte Version der Android Support Bibliothek, aus der einige, der in den nächsten Abschnitten beschriebenen Komponente, in der App verwendet werden.

\subsubsection{Play Services} \label{subsubsec:play-service-libraries}
Zur Ansteuerung der Kamera des Smartphones für den Barcode-Scanner, musste eine Play Services Bibliothek implementiert werden.

\subsubsection{Benutzeroberfläche} \label{subsubsec:ui-libraries}
Die AppCompat Bibliothek, aus der siebten Version der Android Support Bibliothek, bietet Unterstützung für die ActionBar, mit der die Toolbar der App implementiert wurde.
Mit der Design Support Bibliothek können Material Design Komponenten benutzt werden.
Im Fall der App wurde der Floating Action Button verwendet.
RecyclerViews, die die Gruppierung ihrer Listenelemente ermöglichen, wurde mit Hilfe der SectionedRecyclerViewAdapter Bibliothek erreicht.
Für die Implementierung des Barcode-Scanner der App wurde eine Bibliothek vom Entwickler Ravi verwendet und für die Darstellungen der Statistiken im Statistiken-Screen die MPAndroidChart-Bibliothek vom Entwickler Philipp Jahoda.

\subsubsection{Datenmodell} \label{subsubsec:model-libraries}
Mit Hilfe des von Albert Zündorf entwickelten Fulibs wurde das Datenmodell der App generiert.

\subsubsection{Netzwerk} \label{subsubsec:network-libraries}
Zur Kommunikation mit dem REST-Server von ComTec wurde OkHttp verwendet.
Für die Verarbeitung der JSON-Daten wurde Gson benutzt.

\subsubsection{Tests} \label{subsubsec:tests-libraries}
Für allgemeine Tests wurde JUnit verwendet.
Das JMock Framework wurde gebraucht, um den Server für diese Tests zu simulieren.
Zum Testen der Benutzeroberfläche wurde das Espresso-Framework benutzt.
