\documentclass{scrartcl}

\usepackage[ngerman]{babel}
\usepackage[utf8]{inputenc}
\usepackage[hidelinks]{hyperref}
\usepackage{graphicx}

\begin{document}
	\title{CodeCamp: Context Awareness, Entwicklung einer Android-App}
	\author{Adrian Kunz, Jan Bingemann, Johann Feser, Michael Prasil, Sven Starcke}
	\date{Kassel, den \today}
	
	\maketitle
	\vspace*{10ex}
	\tableofcontents
	
	\newpage
	
	\section{Einleitung}
		\paragraph*{Ziele}
		Ziel des Projektes war es, im Rahmen der Veranstaltung \glqq Code Camp 1: Context Awareness\grqq{} eine Android-basierte Anwendung zur Verwaltung und Aufzeichnung der Einkäufe, welche in der ComTec-Küche getätigt werden, zu erstellen.
		
		\paragraph*{Funktionsweise}
		Auf Basis einer Datenbank können Gegenstände angelegt, gekauft und gelöscht werden. Ein dazu benötigter Nutzer wird ebenfalls in der Datenbank gespeichert. Der Zugriff auf die Datenbank erfolgt über eine REST-API, welche alle notwendigen Operationen bereitstellt.
		
		\paragraph*{Inhalt}
		In Abschnitt ~\ref{features} werden die von der App bereitgestellten Features vorgestellt. Abschnitt ~\ref{architecture} erklärt die Anwendungsstruktur aus technischer Sicht, jeweils für Datenmodell (Abschnitt ~\ref{architecture.datamodel}), die Oberflächen-Architektur (Abschnitt ~\ref{architecture.frontend}) und die zugrunde liegende Grundstruktur inklusive Netzwerkverkehr (Abschnitt ~\ref{architecture.backend}).
		
	\section{Andwendung} \label{features}
		\paragraph*{Start}
		Beim Öffnen der Anwendung erscheint eine Login-Maske, sofern sich nicht in den Shared-Preferences des Android-Geräts bereits ein gültiges User-Token und eine passende User-Id befindet. In letzterem Fall wird der Nutzer anhand der Daten aus den Shared-Preferences eingeloggt. Falls keine Anmeldedaten vorhanden sind, wird ein neuer Nutzer auf Basis der in der Oberfläche eingegebenen Anmeldedaten (Name, E-Mail, Adminflag, siehe Abbildung ~\ref{loginscreen}) erstellt. Das User-Token wird für die weitere Verwendung zwischengespeichert.
	
		\begin{figure}[!h]
			\label{loginscreen}
			\centering
			%\includegraphics[scale=0.5]{loginscreen.png}
			\caption{Login-Screen}
		\end{figure}
	
		\paragraph*{Haupt-Screen}
		Nach erfolgreicher Anmeldung erfolgt eine Weiterleitung auf den Main-Screen. Dieser zeigt eine Liste der kaufbaren Gegenstände. Ein Eintrag in Dieser Liste enthält folgende Komponenten:
		
		\begin{itemize}
			\item Name: Name des ausgewählten Gegenstands
			
			\item Preis: Der Preis für einen Gegenstande des ausgewählten Typs
			
			\item Ausgewählte Anzahl: Zeigt an, wie viele Gegenstände des ausgewählten Typs man kaufen möchte. Sie erhöht sich mit jedem Klick auf den Eintrag.
			
			\item verfügbare Anzahl: Zeigt an, wie viele Gegenstände des ausgewählten Typs gekauft werden können. 
		\end{itemize}
	
		Die Liste ist nach der Art der Gegenstände sortiert (Beispiel: Wasser, Saft, \ldots).
		
		\paragraph*{}
		Ist der aktuelle Nutzer als Administrator ausgezeichnet, kann er mit dem +-Knopf in der unteren, rechten Ecke des Bildschirms neue Gegenstände hinzufügen.
		
		\begin{figure}[!h]
			\label{mainscreen}
			\centering
			%\includegraphics[scale=0.5]{mainscreen.png}
			\caption{Haupt-Screen}
		\end{figure}
		
	
	\section{Architektur} \label{architecture}
		\subsection{Datenmodell} \label{architecture.datamodel}
			\paragraph*{}
			Das Datenmodell umfasst drei Klassen: \texttt{Item}, \texttt{User}, \texttt{Purchase}. Die Klassen sind passend zur bereitgestellten Datenbank implementiert und haben die in Abbildung ~\ref{datamodel} gezeigte Struktur.
		
		\begin{figure}[!h]
			\label{datamodel}
			\centering
			%\includegraphics[scale=0.5]{datamodel.png}
			\caption{Datenmodell}
		\end{figure}
		
		
		\subsection{Frontend-Architektur} \label{architecture.frontend}
		
		
		\subsection{Backend-Architektur} \label{architecture.backend}
			\paragraph*{}
			Die Backend-Architektur besteht aus drei Ebenen (siehe Abbildung ~\ref{backendArchitecture}). Die unterste Ebene bildet die Klasse \texttt{OkHttpConnection}, welche über das Interface \texttt{HttpConnection} implementiert ist. Die Klasse ist verantwortlich für den primitiven Netzwerkverkehr per HTTP mit beliebigen Inhalten. Es werden die HTTP-Methoden \texttt{GET}, \texttt{POST}, \texttt{PUT} und \texttt{DELETE} implementiert. Bei fehlerhaften Anfragen (HTTP-Code $\neq$ 2xx), sowie bei fehlerhaften URLs wird eine Exception geworfen.
			
			\paragraph*{}
			Auf der zweiten Ebene befinden sich die Klassen \texttt{KitchenConnection}, \texttt{LocalDataStore} und \texttt{JsonTranslator}.
			
			\begin{itemize}
				\item  \texttt{KitchenConnection} ist verantwortlich für die spezialisierte Kommunikation mit der bereitgestellten REST-API. Die Klasse enthält Attribute für die Basis-URL des REST-Servers, sowie für Schlüssel zur Erzeugung von Nutzern. Alle Funktionen der REST-API werden hier implementiert, indem jede Methode der Klasse einen Pfad der REST-API abdeckt. Die Funktionsnamen entsprechen der Spezifikation der REST-API.
				
				\item \texttt{LocalDataStore} ist verantwortlich für das Zwischenspeichern der User, Items und Purchases. Dies ermöglicht schnelleren Zugriff auf die jeweiligen Objekte.
				
				\item \texttt{JsonTranslator} ist verantwortlich für die bidirektionale Umwandlung zwischen Objekten des Datenmodells und JSON-Strings. Die verwendete Bibliothek für die Umwandlung ist Google's \texttt{GSON}.
			\end{itemize}
		
			\paragraph*{}
			Die Dritte Ebene beinhaltet nur die Klasse \texttt{KitchenManager}. Sie ist der Einstiegspunkt des Backend und stellt so alle Operationen zur Verfügung, welche die Oberfläche benötigt.
		
			\begin{figure}[!h]
				\centering
				\label{backendArchitecture}
				\includegraphics[scale=0.5]{./figures/classStructure.png}
				\caption{Backend Architektur}
			\end{figure}

			\paragraph*{Tests}
			Alle Backend-Klassen sind durch Unit-, Integrations- und Akzeptanz-Tests ausreichend getestet.
	
	\section{Anhang: Bibliotheken} \label{bib}
		
\end{document}